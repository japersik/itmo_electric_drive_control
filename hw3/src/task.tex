\section{Цель работы}
\begin{enumerate}
    \item Рассчитать коэффициент датчика момента из условия поддержания номинального
    момента при величине напряжения задания 10В.
    \item Параметры ПИ-регулятора момента из условия настройки системы на
    технический оптимум.
    \item Реализовать математическую модель контура в пакете MATLAB.
    \item Снять реакции $M(t)$, $U_y(t)$, $\varepsilon(t)$ на скачкообразное изменение задающего воздействия
    при нулевых начальных условиях, исключив влияние эл. /мех. связи. Определить
    параметры $M(t)$: время первого согласования $t_\text{p1}$, перерегулирование, время переходного
    процесса $t_\text{п}$ и сравнить с параметрами эталонной кривой.
    \item Выполнить программу п.4 c учетом эл./мех. связи.
\end{enumerate}



\section{Данные варианта}
\begin{itemize}
    \item Nпп: 14
    \item $\omega_\text{0ном}$: 706 (1/c)
    \item $M_\text{ном}$: 13.7 (Нм)
    \item $M_\text{п}$: 24.7 (Нм)
    \item $J_\text{1}$: 0.008 (кгм$^2$)
    \item $J_\text{2}$: 0.0025 (кгм$^2$)
    \item $C_\text{12}$: 300
    \item $T_\text{э}$: 50 (мс)
    \item $T_\text{пр}$: 10 (мс)
    \item $K_\text{пр}$: 15
    \item $M_\text{с1}$: 10 (Нм)
    \item $M_\text{с2}$: 3.7 (Нм)
\end{itemize}
\newpage