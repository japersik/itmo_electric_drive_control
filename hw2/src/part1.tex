\section{Марериалы работы}
\subsection{Расчет переходных процессов}
Так как нам известна электромагнитная постоянная, рассчитаем электромеханическую постоянную и статическую жесткость.
\begin{gather*}
    \beta = \frac{M_\text{п}}{\omega_\text{ном}}=0.035\\
    T_M = \frac{J_1+J_2}{\beta} = 0.3001
\end{gather*}
Из отношения $4T_\text{э}<T_\text{М}$ имеем два вещественных корня передаточной функции.
Составим характеристическое уравнение и найдем его корни:
\begin{gather*}
    T_\text{э}T_\text{М}\lambda^2+ T_\text{М}\lambda + 1 = 0\\
    \lambda_1 =   -4.2242\\
    \lambda_2 =    -15.7758
\end{gather*}
Определим время переходного процесса:
\begin{gather*}
    t_\text{п} = \frac{3}{|\lambda_1|} = 0.7102
\end{gather*}
\newpage
\subsection{Одномассовый механизм}
Запишем математическую модель ДПТ с одномассовым механизмом в виде системы дифференциальных уравнений

\begin{gather*}
    \begin{cases}
        \dot{M} = \frac{\beta}{T_\text{э}}\omega_0-\frac{1}{T_\text{э}}M-\frac{\beta}{T_\text{э}}\omega_1\\
        \dot{\omega}_1 = \frac{M}{\beta T_M}-\frac{M_c}{\beta T_M}
    \end{cases}
\end{gather*}
Преобразуем уравнения в систему вида вход-состояние-выход в матричной форме
\begin{gather*}
    \begin{bmatrix}
        \dot{M} \\
        \dot{\omega}_1
    \end{bmatrix}
    =
    \begin{bmatrix}
        -\frac{1}{T_\text{э}} & -\frac{\beta}{T_\text{э}}\\
        -\frac{1}{\beta T_\text{M}} & 0
    \end{bmatrix}
    \begin{bmatrix}
        M \\
        \omega_1
    \end{bmatrix}+
    \begin{bmatrix}
        \frac{\beta}{T_\text{э}}&0\\
        0&\frac{1}{\beta T_M}\\
    \end{bmatrix}
    \begin{bmatrix}
        \omega_0\\
        M_c
    \end{bmatrix}
\end{gather*}
Проведем моделирование системы при $\omega_0 = 0$, $M_c = 0.1M_\text{ном}$
\begin{figure}[!h]
    \centering
    \begin{minipage}{0.5\textwidth}
        \centering
        \includegraphics[width = \textwidth]{img/task12_M}
        \label{fig:img/task12_M}
    \end{minipage}%
    \begin{minipage}{0.5\textwidth}
        \centering
        \includegraphics[width = \textwidth]{img/task12_omega1}
        \label{fig:img/task12_omega1}
    \end{minipage}%
    \caption{Результат моделирования системы при $\omega_0 = 0$, $M_c = 0.1M_\text{ном}$}
\end{figure}

Проведем моделирование системы при $\omega_0 = 0.1\omega_\text{ном}$, $M_c = 0$
\begin{figure}[!h]
    \centering
    \begin{minipage}{0.5\textwidth}
        \centering
        \includegraphics[width = \textwidth]{img/task11_M}
        \label{fig:img/task11_M}
    \end{minipage}%
    \begin{minipage}{0.5\textwidth}
        \centering
        \includegraphics[width = \textwidth]{img/task11_omega1}
        \label{fig:img/task11_omega1}
    \end{minipage}%
    \caption{Результат моделирования системы при $\omega_0 = 0.1\omega_\text{ном}$, $M_c = 0$}
\end{figure}

\subsection{Двухмассовый механизм}
Запишем математическую модель ДПТ с двухмассовым механизмом в виде системы дифференциальных уравнений
\begin{gather*}
    \begin{cases}
        \dot{M} = \frac{\beta}{T_\text{э}}\omega_0-\frac{\beta}{T_\text{э}}M-\frac{\beta}{T_\text{э}}\omega_1\\
        \dot{\omega}_1 = \frac{M}{J_1}-\frac{M_{12}}{J_1}-\frac{M_{c1}}{J_1}\\
        \dot{M}_{12} = C_{12}\omega_1 -C_{12}-\omega_2\\
        \dot{\omega}_2 = \frac{M_2}{J_2}-\frac{M_{c2}}{J_2}
    \end{cases}
\end{gather*}
Преобразуем уравнения в систему вида вход-состояние-выход в матричной форме
\begin{gather*}
    \begin{bmatrix}
        \dot{M} \\
        \dot{\omega}_1\\
        \dot{M}_{12} \\
        \dot{\omega}_2
    \end{bmatrix}
    =
    \begin{bmatrix}
        -\frac{\beta}{T_\text{э}}&-\frac{\beta}{T_\text{э}}&0&0\\
        \frac{1}{J_1} 0 -\frac{1}{J_1} 0 \\
        0&C_{12}&0&-C_{12}\\
        0&0&J_2&0\\
    \end{bmatrix}
    \begin{bmatrix}
        M \\
        \omega_1\\
        M_{12} \\
        \omega_2
    \end{bmatrix}+
    \begin{bmatrix}
        \frac{\beta}{T_\text{э}}&0&0\\
        0 & -\frac{1}{J_1}&0\\
        0&0&0\\
        0&0&\frac{1}{J_2}\\
    \end{bmatrix}
    \begin{bmatrix}
        \omega_0\\
        M_{c1}\\
        M_{c2}\\
    \end{bmatrix}
\end{gather*}

Проведем моделирование системы при $\omega_0 = 0$, $M_c = 0.1M_\text{ном}$
\begin{figure}[!h]
    \centering
    \begin{minipage}{0.5\textwidth}
        \centering
        \includegraphics[width = \textwidth]{img/task21_omega1}
        \label{fig:img/task21_omega1}
    \end{minipage}%
    \begin{minipage}{0.5\textwidth}
        \centering
        \includegraphics[width = \textwidth]{img/task21_omega2}
        \label{fig:img/task21_omega2}
    \end{minipage}
    \begin{minipage}{0.5\textwidth}
        \centering
        \includegraphics[width = \textwidth]{img/task21_M}
        \label{fig:img/task21_M}
    \end{minipage}%
    \begin{minipage}{0.5\textwidth}
        \centering
        \includegraphics[width = \textwidth]{img/task21_M12}
        \label{fig:img/task21_M12}
    \end{minipage}%
    \caption{Результат моделирования системы при $\omega_0 = 0$, $M_c = 0.1M_\text{ном}$}
\end{figure}

 \newpage
Проведем моделирование системы при $\omega_0 = 0.1\omega_\text{ном}$, $M_c = 0$
\begin{figure}[!h]
    \centering
    \begin{minipage}{0.5\textwidth}
        \centering
        \includegraphics[width = \textwidth]{img/task22_omega1}
        \label{fig:img/task22_omega1}
    \end{minipage}%
    \begin{minipage}{0.5\textwidth}
        \centering
        \includegraphics[width = \textwidth]{img/task22_omega2}
        \label{fig:img/task22_omega2}
    \end{minipage}
    \begin{minipage}{0.5\textwidth}
        \centering
        \includegraphics[width = \textwidth]{img/task22_M}
        \label{fig:img/task22_M}
    \end{minipage}%
    \begin{minipage}{0.5\textwidth}
        \centering
        \includegraphics[width = \textwidth]{img/task22_M12}
        \label{fig:img/task22_M12}
    \end{minipage}%
    \caption{Результат моделирования системы при $\omega_0 = 0$, $M_c = 0.1M_\text{ном}$}
\end{figure}
