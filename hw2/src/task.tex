\section{Цель работы}
\begin{enumerate}
    \item Для заданного варианта рассчитать характер и время электромеханических
    переходных процессов.
    \item Записать и реализовать среде MATLAB векторно-матричную модель двигателя с
    одномассовым механизмом.
    \item Получить графики зависимостей $M(t)$ и $\omega_1(t)$ для случаев
     \begin{enumerate}
              \item реакция на скачок управляющего воздействия от 0 до 0.1$\omega_{\text{0ном}}$  при нулевом моменте
              нагрузки $M_c=0$;
              \item реакция на скачок момента нагрузки  0 до 0.1$M_{\text{0ном}}$при нулевом управляющем
              воздействии.
    \end{enumerate}
    \item Определить по графикам начальные и принужденные значения скорости и момента и
    время переходного процесса и сравнить с расчетными.
    \item Записать и реализовать среде MATLAB векторно-матричную модель ЭМП с
    двухмассовым механизмом.
    \item Получить графики зависимостей $M(t)$, $M_{12}(t)$ , $\omega_1(t)$, $\omega_2(t)$ для случаев 3а и 3б.
    \item Сформировать выводы по результатам сравнения характеристик двигателя с
    одномассовым и двухмассовым механизмами.
\end{enumerate}



\section{Данные варианта}
\begin{itemize}
    \item Nпп: 14
    \item $\omega_\text{0ном}$: 706 (1/c)
    \item $M_\text{ном}$: 13.7 (Нм)
    \item $M_\text{п}$: 24.7 (Нм)
    \item $J_\text{1}$: 0.008 (кгм$^2$)
    \item $J_\text{2}$: 0.0025 (кгм$^2$)
    \item $C_\text{12}$: 300
    \item $T_\text{э}$: 50 (мс)
    \item $T_\text{пр}$: 10 (мс)
    \item $K_\text{пр}$: 15
    \item $M_\text{с1}$: 10 (Нм)
    \item $M_\text{с2}$: 3.7 (Нм)
\end{itemize}
\newpage