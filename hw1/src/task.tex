\section{Цель работы}
\begin{itemize}
    \item Реализовать двухмассовую модель механизма в уравнениях состояния в среде MATLAB
    \item Снять на математической модели реакцию механизма на скачок момента $M$, величиной $0.1M_\text{ном}$.
    Вывести графики $\omega_1(t)$, $\omega_2(t)$, $M_{12}(t)$.
    \item Сравнить параметры полученных кривых с расчетными.
    Сделать выводы о результате сравнения расчетных характеристик с экспериментальными.
\end{itemize}



\section{Данные варианта}
\begin{itemize}
    \item Nпп: 14
    \item $\omega_\text{0ном}$: 706 (1/c)
    \item $M_\text{ном}$: 13.7 (Нм)
    \item $M_\text{п}$: 24.7 (Нм)
    \item $J_\text{1}$: 0.008 (кгм$^2$)
    \item $J_\text{2}$: 0.0025 (кгм$^2$)
    \item $C_\text{12}$: 300
    \item $T_\text{э}$: 50 (мс)
    \item $T_\text{пр}$: 10 (мс)
    \item $K_\text{пр}$: 15
    \item $M_\text{с1}$: 10 (Нм)
    \item $M_\text{с2}$: 3.7 (Нм)
\end{itemize}
\newpage